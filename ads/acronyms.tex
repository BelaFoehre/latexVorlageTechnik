\addchap{\langabkverz}
%nur verwendete Akronyme werden letztlich im Abkürzungsverzeichnis des Dokuments angezeigt
%Verwendung: 
%		\ac{Abk.}   --> fügt die Abkürzung ein, beim ersten Aufruf wird zusätzlich automatisch die ausgeschriebene Version davor eingefügt bzw. in einer Fußnote (hierfür muss in header.tex \usepackage[printonlyused,footnote]{acronym} stehen) dargestellt
%		\acs{Abk.}   -->  fügt die Abkürzung ein
%		\acf{Abk.}   --> fügt die Abkürzung UND die Erklärung ein
%		\acl{Abk.}   --> fügt nur die Erklärung ein
%		\acp{Abk.}  --> gibt Plural aus (angefügtes 's'); das zusätzliche 'p' funktioniert auch bei obigen Befehlen
%	siehe auch: http://golatex.de/wiki/%5Cacronym
%	
\begin{acronym}[YTMMM]
\setlength{\itemsep}{-\parsep}

\acro{AGPL}{Affero GNU General Public License}
\acro{WSN}{Wireless Sensor Network}
\acro{MANET}{Mobile wireless Ad-hoc NETwork}
\acro{MAC}{Multiple Access Control}
\acro{QoS}{Quality of Service}
\acro{DSR}{Dynamic Source Routing}
\acro{API}{Application Programming Interface}
\acro{WYSIWYG}{What You See Is What You Get}
\acro{HTML}{HyperText Markup Language}
\acro{SaaS}{Software-as-a-Service}
\acro{PaaS}{Platform-as-a-Service}
\acro{IaaS}{Infrastructure-as-a-Service}
\acro{CaaS}{Container-as-a-Service}
\acro{BaaS}{Backend-as-a-Service}
\acro{FaaS}{Functions-as-a-Service}
\acro{XaaS}{Everything-as-a-Service}
\acro{AWS}{Amazon Web Services}
\acro{NIST}{National Institute of Standards and Technology}
\acro{VPN}{Virtual Private Network}
\acro{REST}{Representational State Transfer}
\acro{JSON}{JavaScript Object Notation}
\acro{ECS}{Elastic Container Services}
\acro{VPC}{Virtual Private Cloud}
\acro{ECR}{Elastic Container Registry}
\acro{CI/CD}{Continuous Integration / Continuous Deployment}
\acro{CIA}{Consistency, Integrity and Availability}
\acro{VM}{Virtuelle Maschine}
\acrodefplural{VM}[VMs]{Virtuellen Maschinen}

\end{acronym}
