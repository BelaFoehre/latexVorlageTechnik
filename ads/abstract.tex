%!TEX root = ../dokumentation.tex

\pagestyle{empty}

\iflang{de}{%
% Dieser deutsche Teil wird nur angezeigt, wenn die Sprache auf Deutsch eingestellt ist.
\renewcommand{\abstractname}{\langabstract} % Text für Überschrift

% \begin{otherlanguage}{english} % auskommentieren, wenn Abstract auf Deutsch sein soll
\begin{abstract}

Um eine lokale Anwendung unternehmensintern weltweit ohne Installation verfügbar zu machen, ist eine Transformation dieser in die Cloud ein mögliches Konzept, um dies umzusetzen. Eine solche Cloud Migration kann sowohl Vor- als auch Nachteile mit sich bringen, die in der Arbeit untersucht werden sollen.

Dazu werden die Entwicklung und aktuelle Trends des Cloud Computing untersucht. Desweiteren wird in dieser Arbeit eine Financial Management Anwendung, die bisher aud einem lokalen System ausgeführt wird, prototypisch zu einer cloudfähigen Anwendung entwickelt und in der Cloud ausgeführt, um notwendige Veränderungen an der Anwendungsarchitektur herauszuarbeiten. Für eine Vergleichbarkeit der Anwendungsversionen werden unter anderem anhand einer Use-Case Analyse die Anforderungen für die migrierte Anwendung ermittelt.

Ziel ist es, die Vor- und Nachteile aus der Analyse herauszuarbeiten und einen Einblick in die tatsächliche Durchführung der Anwendungstransformation zu geben.

\end{abstract}
% \end{otherlanguage} % auskommentieren, wenn Abstract auf Deutsch sein soll
}



\iflang{en}{%
% Dieser englische Teil wird nur angezeigt, wenn die Sprache auf Englisch eingestellt ist.
\renewcommand{\abstractname}{\langabstract} % Text für Überschrift

\begin{abstract}
An abstract is a brief summary of a research article, thesis, review, conference proceeding or any in-depth analysis of a particular subject or discipline, and is often used to help the reader quickly ascertain the paper's purpose. When used, an abstract always appears at the beginning of a manuscript, acting as the point-of-entry for any given scientific paper or patent application. Abstracting and indexing services for various academic disciplines are aimed at compiling a body of literature for that particular subject.

The terms précis or synopsis are used in some publications to refer to the same thing that other publications might call an ``abstract''. In ``management'' reports, an executive summary usually contains more information (and often more sensitive information) than the abstract does.

Quelle: \url{http://en.wikipedia.org/wiki/Abstract_(summary)}

\end{abstract}
}