%!TEX root = ../dokumentation.tex

%
% vorher in Konsole folgendes aufrufen:
%	makeglossaries makeglossaries dokumentation.acn && makeglossaries dokumentation.glo
%

%
% Glossareintraege --> referenz, name, beschreibung
% Aufruf mit \gls{...}
%
\newglossaryentry{Glossareintrag}{name={Glossareintrag},plural={Glossareinträge},description={Ein Glossar beschreibt verschiedenste Dinge in kurzen Worten}}
\newglossaryentry{IntelliJ}{name={IntelliJ},description={Eine Programmierumgebung für Java Projekte}}
\newglossaryentry{Spring Boot}{name={Spring Boot},description={\glqq{Spring Boot makes it easy to create stand-alone, production-grade Spring based Applications that you can \lq{just run}\rq{}.}\grqq{}}\cite[]{Springboot}}
\newglossaryentry{Terraform}{name={Terraform},description={folgt\dots}}
\newglossaryentry{Spring}{name={Spring},description={\glqq{The Spring Framework provides a comprehensive programming and configuration model for modern Java-based enterprise applications - on any kind of deployment platform.}\grqq{}}\cite[]{Spring}}
\newglossaryentry{Box}{name={Box},description={Box bietet einen online Cloud-Speicher für Unternehmen.}}
\newglossaryentry{GitHub}{name={GitHub},description={Git ist ein Tool zur Versionierungsverwaltung von Software-Projekten. GitHub dient der Visualisierung und einfacheren Bedienung von Git-Repositories. \cite[Vgl.][]{t3n2021}}}
\newglossaryentry{Repository}{name={Repository},description={\glqq{Ein Repository (oder kurz Repo) kann als Projektverzeichnis verstanden werden. Die Dateien für ein Software-Projekt werden in einem Repository abgelegt.}\grqq{} \cite[Vgl.][]{t3n2021}}}
\newglossaryentry{Fargate}{name={Fargate},description={\dots}}
\newglossaryentry{CodePipeline}{name={CodePipeline},description={\dots}}