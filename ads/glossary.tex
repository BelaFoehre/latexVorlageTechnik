%!TEX root = ../dokumentation.tex

%
% vorher in Konsole folgendes aufrufen:
%	makeglossaries makeglossaries dokumentation.acn && makeglossaries dokumentation.glo
%

%
% Glossareintraege --> referenz, name, beschreibung
% Aufruf mit \gls{...}
%
\newglossaryentry{Glossareintrag}{name={Glossareintrag},plural={Glossareinträge},description={Ein Glossar beschreibt verschiedenste Dinge in kurzen Worten}}
\newglossaryentry{IntelliJ}{name={IntelliJ},description={Eine Programmierumgebung für Java Projekte}}
\newglossaryentry{Spring Boot}{name={Spring Boot},description={''Spring Boot makes it easy to create stand-alone, production-grade Spring based Applications that you can \lq{just run}\rq{}.''}\cite[]{Springboot}}
\newglossaryentry{Terraform}{name={Terraform},description={''Terraform is an infrastructure as code (IaC) tool that allows you to build, change, and version infrastructure safely and efficiently. This includes both low-level components like compute instances, storage, and networking, as well as high-level components like DNS entries and SaaS features.'' \cite[][]{HashiCorp}}}
\newglossaryentry{Spring}{name={Spring},description={''The Spring Framework provides a comprehensive programming and configuration model for modern Java-based enterprise applications - on any kind of deployment platform.''}\cite[]{Spring}}
\newglossaryentry{Box}{name={Box},description={Box bietet einen online Cloud-Speicher für Unternehmen, kann aber auch von Privatanwendern verwendet werden. Funktionalität vergleichbar zur bekannteren Dropbox-Plattform.}}
\newglossaryentry{GitHub}{name={GitHub},description={Git ist ein Tool zur Versionierungsverwaltung von Software-Projekten. GitHub dient der Visualisierung und einfacheren Bedienung von Git-Repositories. \cite[Vgl.][]{t3n2021}}}
\newglossaryentry{Repository}{name={Repository},description={''Ein Repository (oder kurz Repo) kann als Projektverzeichnis verstanden werden. Die Dateien für ein Software-Projekt werden in einem Repository abgelegt.'' \cite[Vgl.][]{t3n2021}}}
\newglossaryentry{Fargate}{name={Fargate},description={''AWS Fargate is a serverless, pay-as-you-go compute engine that lets you focus on building applications without managing servers. AWS Fargate is compatible with both Amazon Elastic Container Service (ECS) and Amazon Elastic Kubernetes Service (EKS).'' \cite[][]{AWSFargate}}}
\newglossaryentry{CodePipeline}{name={CodePipeline},description={''AWS CodePipeline is a fully managed continuous delivery service that helps you automate your release pipelines for fast and reliable application and infrastructure updates. CodePipeline automates the build, test, and deploy phases of your release process every time there is a code change, based on the release model you define. This enables you to rapidly and reliably deliver features and updates. You can easily integrate AWS CodePipeline with third-party services such as GitHub or with your own custom plugin. With AWS CodePipeline, you only pay for what you use.'' \cite[][]{AWSCodePipeline}}}
\newglossaryentry{Multi-Tenancy}{name={Multi-Tenancy},description={Auf deutsch: Mehrbenutzerfähigkeit\newline Multi-tenancy ist eine Anforderung an eine Anwendung, die von mehreren Benutzern gleichzeitig genutzt werden kann/soll (z. B. in der Cloud), ohne dass diese sich in die Quere kommen und z. B. durch gleichzeitige Nutzung fehlerhafte Daten entstehen oder falsche Nutzer Daten erhalten}}
\newglossaryentry{Timesheet}{name={Timesheet},description={Arbeitsnachweis in form einer Exceldatei, in der Mitarbeiter ihre Tätigkeit und die benötigte Zeitdauer bezogen auf einen Kundenauftrag für den jeweiligen Tag erfassen.},plural={Timesheets}}
\newglossaryentry{Apache POI}{name={Apache POI},description={''The Apache POI Project's mission is to create and maintain Java APIs for manipulating various file formats based upon the Office Open XML standards (OOXML) and Microsoft's OLE 2 Compound Document format (OLE2). In short, you can read and write MS Excel files using Java. In addition, you can read and write MS Word and MS PowerPoint files using Java. Apache POI is your Java Excel solution (for Excel 97-2008).'' \cite[][]{ApachePOI}}}