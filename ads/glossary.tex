%!TEX root = ../dokumentation.tex

%
% vorher in Konsole folgendes aufrufen:
%	makeglossaries makeglossaries dokumentation.acn && makeglossaries dokumentation.glo
%

%
% Glossareintraege --> referenz, name, beschreibung
% Aufruf mit \gls{...}
%
\newglossaryentry{Glossareintrag}{name={Glossareintrag},plural={Glossareinträge},description={Ein Glossar beschreibt verschiedenste Dinge in kurzen Worten}}
\newglossaryentry{IntelliJ}{name={IntelliJ},description={Eine Programmierumgebung für Java Projekte}}
\newglossaryentry{Springboot}{name={Springboot},description={\glqq{Spring Boot makes it easy to create stand-alone, production-grade Spring based Applications that you can \lq{just run}\rq{}.}\grqq{}}\cite[]{Springboot}}
\newglossaryentry{Terraform}{name={Terraform},description={folgt\dots}}
\newglossaryentry{Spring}{name={Spring},description={\glqq{The Spring Framework provides a comprehensive programming and configuration model for modern Java-based enterprise applications - on any kind of deployment platform.}\grqq{}}\cite[]{Spring}}