\section{Kritische Auswahl der Forschungsmethoden}
\label{sec:auswahl_forschungsmethoden}

% Zielsetzung
% Literaturrecherche
% -hierzu zunächst aktuellen Stand der Forschung untersuchen
%  ->Thema x,y,z
% -Methodik beschreiben
% Anforderungsanalyse
% -non-functional Requirements
% -was muss beachtet werden für Cloud Anwendung
% -Anforderungen Cloud
% Use-Case Analyse
% -Werden Use-Cases immer noch erfüllt?
%  ->zufriedenstellend?
% -Architekturelle Änderungen (Client -> Web)
% Prototyping
% Evaluation

Um die Forschungsfrage dieser Arbeit zu entwickeln, wird zunächst der aktuelle Stand der wissenschaftlichen Forschung zu dem Thema Cloud Computing und Cloud Migration untersucht. Besonderer Fokus wird hierbei auf die Untersuchung der Herausforderungen der Cloud Migration gelegt, bevor die Durchführung eines Migrationsansatzes betrachtet wird.

Zu diesem Zweck wird eine Literaturrecherche nach Döring/Bortz 2016 durchgeführt. Dazu werden einige Begriffe zur systematischen Durchsuchung wissenschaftlicher Datenbanken festgelegt \cite[Vgl.][S. 158]{Doering2016}. Als Suchmaschine wurde in diesem Fall hauptsächlich Google Scholar genutzt, da hierüber eine Vielzahl verschiedener Datenbanken angebunden sind. Darüber hinaus wurde zusätzlich noch explizit die Datenbank des Springer Verlages verwendet, vereinzelt auch noch die Datenbanken von arXiv, IEEE oder ACM.

Zur Recherche wurden die folgenden Begriffe verwendet: ''Cloud Migration'', ''Cloud Computing'', ''Cloud'', ''Herausforderungen Cloud Migration''

Diese werden als \textbf{''primäre Suchbegriffe''} \cite[S. 158]{Doering2016} festgelegt. Die Suche wurde darüber hinaus nach dem Schneeballsystem durchgeführt, was bedeutet, dass die Quellen der verwendeten Veröffentlichungen zur weiteren Recherche verwendet werden \cite[Vgl.][S. 160]{Doering2016}. Aus den Quellen werden außerdem weitere Begriffe als \textbf{''sekundäre Suchbegriffe''} \cite[S. 158]{Doering2016} festgelegt und zur erweiterten Suche verwendet. Zu diesen gehören unter anderem ''Migration zu PaaS'', ''Cloud-Native'' und ''\textit{Refactoring}''.

Zur Durchführung des praktischen Teils dieser Arbeit wird darüber hinaus eine Anforderungsanalyse mit Use-Case Modellierung durchgeführt, um die Anforderungen an das anschließend folgende \textit{Prototyping} nach Wilde/Hess (2007) und Heinrich (2011) zu erarbeiten. Mithilfe des Prototypen soll schließlich die Cloud Migration durchgeführt und das Ergebnis untersucht werden. Detaillierte Ausführungen zur Forschung im praktischen Teil folgen in Kapitel \ref{chap:forschungsmethoden}.