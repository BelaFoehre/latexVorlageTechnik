\section{Geschäftlicher Kontext}

In dieser Arbeit wird die Migration einer lokal laufenden Anwendung aus dem Projektmanagement zur Rechnungsstellung in die Cloud untersucht.
Zweck der Anwendung ist das Einsammeln sogenannter Timesheets von Projektmitarbeitern aus einem Cloud Spiecher. Diese Timesheets dokumentieren,
wie viel Arbeitszeit ein Mitarbeiter für die verschiedenen Aufgaben in einem Projekt verbracht hat um daraus präzise Rechnungen erstellen zu können.

Das Tool wird bereits heute von mehreren Personen im Projekt verwendet und soll in Zukunft auch für andere Projekte eingesetzt werden. Aus diesem
Grund wird die Migration dieser Anwendung in die Cloud untersucht.

Die durchgeführte Migration soll die Anwendung daher global verfügbar machen und vorallem den Aufwand aber auch gegebenenfalls Kosten reduzieren.
Diese Anwendung muss bisher vom Anwender aus GitHub heruntergeladen und lokal ausgeführt werden.
Zur lokalen Ausführung sind einige Anpassungen in der Konfigurationsdatei notwendig, wenn sich das ausführende System ändert.
Darüber hinaus muss die Anwendung nach jeder Aktualisierung manuell erneut heruntergeladen werden.
Das zur Verfügung stellen in der Cloud soll die Benutzung der Anwendung erleichtern und es auf lange Sicht ermöglichen, dass diese auch außerhalb der Projektumgebung verwendet werden kann.