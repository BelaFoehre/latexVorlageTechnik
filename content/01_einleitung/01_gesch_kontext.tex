\section{Geschäftlicher Kontext}

% Beschreibung aus Businesssicht
In dieser Arbeit wird die Migration einer lokal laufenden Anwendung aus dem Projektmanagement zur Rechnungsstellung in die Cloud untersucht. Zweck der Anwendung ist das Einsammeln von Timesheets von Projektmitarbeitern aus einem Cloud Speicher. Diese Timesheets dokumentieren, wie viel Arbeitszeit ein Mitarbeiter für die verschiedenen Aufgaben in einem Projekt verbracht hat um daraus präzise Rechnungen erstellen zu können.

% Beschreibung aus technischer Sicht
% Gründe/Ziele
% -wird von mehreren eingesetzt
Das Tool wird bereits heute von mehreren Personen im Projekt verwendet und soll in Zukunft auch für andere Projekte eingesetzt werden. Aus diesem Grund wird die Migration dieser Anwendung in die Cloud untersucht.
% -Automatisierung
Darüber hinaus soll durch die Cloud Migration ermöglicht werden, dass einige Prozesse automatisiert werden.

% Herausforderungen
% -Konfiguration
% -Client -> Webanwendung
% -Änderung für non functional Requirements

Die durchgeführte Migration soll die Anwendung daher global verfügbar machen und den Aufwand der Anwendungsausführung verringern. Diese Anwendung muss bisher vom Anwender aus GitHub heruntergeladen und lokal ausgeführt werden. Zur lokalen Ausführung sind einige Anpassungen in Konfigurationsdateien notwendig, da sich das ausführende System ändern kann. Darüber hinaus muss die Anwendung nach jeder Aktualisierung manuell erneut heruntergeladen werden. Das zur Verfügung stellen in der Cloud soll die Benutzung der Anwendung erleichtern und es auf lange Sicht ermöglichen, dass diese auch außerhalb der Projektumgebung von einem größeren Anwenderbereich mit unterschiedlichen IT-Kenntnissen verwendet werden kann.