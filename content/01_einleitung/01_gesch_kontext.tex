\section{Geschäftlicher Kontext dieser Arbeit}

% Beschreibung aus Businesssicht
In dieser Arbeit wird die Migration einer lokal laufenden Anwendung aus dem Projektmanagement zur Rechnungsstellung in die Cloud untersucht. Zweck der Anwendung ist das Einsammeln von \textit{\glspl{Timesheet}} der Projektmitarbeiter aus einem Cloud Speicher. \\Diese \textit{\glspl{Timesheet}} dokumentieren, wie viel Arbeitszeit ein Mitarbeiter für die verschiedenen Aufgaben in dem Projekt verbracht hat um daraus Rechnungen erstellen zu können.

% Beschreibung aus technischer Sicht
% Gründe/Ziele
% -wird von mehreren eingesetzt
Das Tool wird bereits heute von mehreren Personen im Projekt verwendet und soll in Zukunft auch für andere Projekte eingesetzt werden. Aus diesem Grund wird die Migration dieser Anwendung in die Cloud untersucht.
% -Automatisierung
Hierdurch soll auch untersucht werden, wie sich einige Herausforderungen, wie zum Beispiel der Prozess zur Aktualisierung der Anwendung, automatisieren und verbessern lassen sowie die Notwendigkeit der Installation durch den Anwender entfallen kann. Die Migration in die Cloud soll es also ermöglichen die Anwendung global verfügbar zu machen und den Aufwand der Anwendungsausführung sowie der Installation zu verringern.
% Herausforderungen
% -Konfiguration
% -Client -> Webanwendung
% -Änderung für non functional Requirements

Bisher muss die Anwendung aus GitHub heruntergeladen und lokal ausgeführt werden. Zur lokalen Ausführung sind einige Anpassungen in Konfigurationsdateien notwendig, da die Anwendung stark an das ausführende System gekoppelt ist. Darüber hinaus muss die Anwendung nach jeder Aktualisierung manuell erneut heruntergeladen und die Konfigurationsdatei angepasst werden. Das Bereitstellen in der Cloud soll die Benutzung der Anwendung erleichtern und es auf lange Sicht ermöglichen, dass diese auch außerhalb der Projektumgebung von einem größeren Anwenderbereich mit unterschiedlichen IT-Kenntnissen verwendet werden kann.