%!TEX root = ../../dokumentation.tex

\chapter{Einleitung}
% Erste Erwähnung eines Akronyms wird als Fußnote angezeigt. Jede weitere wird
% nur verlinkt: \acf{AGPL}. % \cite{fsf:2007}

% Verweise auf das Glossar: \gls{Glossareintrag}, \glspl{Glossareintrag}

% Motivation
% -Cloud Vorteile
% -> bestehende Anwendungen in Cloud migrieren
Steigende Nutzerzahlen und zunehmende Digitalisierung bringen nach Hentschel und Leyh (2018) einen wachsenden Bedarf an Rechenleistung und immer höhere Anforderungen an Informationssysteme mit sich, denen klassische Modelle der Datenverarbeitung nicht mehr gerecht werden können. Cloud Computing dagegen ist durch Charakteristika wie der Zurverfügungstellung von Infrastruktur für größere Gruppen und Elastizität \cite[Vgl.][S. 2]{Mell2011} in der Lage diese Anforderungen zu erfüllen \cite[Vgl.][S. 6]{Reinheimer2018}.

% Problemstellung
% -Wie lassen sich bei bestehenden Anwendungen die Vorteile von Cloud Ausnutzen?
% -Veränderung Architektur?
%  -zwingend? (konsumieren der Anwendung: rich Client -> Webanwendung)
%  -refactor/rebuild?
%  -Zielarchitektur
Der zuvor gezeigte, anhaltende Trend Anwendungen in der Cloud laufen zu lassen führt auch dazu, dass legacy Anwendungen nach und nach in die Cloud Migriert werden. Nach Kazanavičius, et al. (2019) stellt die Wahl der zum Unternehmen passenden Migrationsmethode eine schwierige Aufgabe dar. Die erste Frage die hier beantwortet werden sollte sei demnach "refactor or rebuild?" \cite[Vgl.][S. 4]{Kazanavicius2019}. Darüber hinaus ergeben sich auch Herausforderungen wie die Frage nach einem adäquaten Architekturdesign für die Cloud \cite[Vgl.][S. 14]{Pahl} und der Untersuchung potenzieller Vor- und Nachteile
von zum Beispiel einer Microservice-Architektur und der Untersuchung möglicher Alternativen \cite[Vgl.][S. 3]{Carrasco2018}.

% Zielsetzung
% -Notwendigkeit
% -Vor- / Nachteile -> Vorteile Cloud
% -Zielarchitektur
Aus der zuvor erarbeiteten Problemstellung ergibt sich als Ziel dieser Arbeit für eine Unternehmensinterne, lokale Anwendung herauszuarbeiten, wie die Migration diese in die Cloud vollzogen werden kann, welche Veränderungen in der Architektur und in Bezug auf die Benutzung vorgenommen werden müssen und in welchen dieser Aufgaben sich größere Herausforderungen befinden.

Explizit nicht betrachtet werden in dieser Arbeit die folgenden Aspekte:
\begin{itemize}
\item erstens
\item zweitens
\item drittens
\end{itemize}

\pagebreak

\section{Geschäftlicher Kontext}

In dieser Arbeit wird die Migration einer lokal laufenden Anwendung aus dem Projektmanagement zur Rechnungsstellung in die Cloud untersucht.
Zweck der Anwendung ist das Einsammeln sogenannter Timesheets von Projektmitarbeitern aus einem Cloud Spiecher. Diese Timesheets dokumentieren,
wie viel Arbeitszeit ein Mitarbeiter für die verschiedenen Aufgaben in einem Projekt verbracht hat um daraus präzise Rechnungen erstellen zu können.

Das Tool wird bereits heute von mehreren Personen im Projekt verwendet und soll in Zukunft auch für andere Projekte eingesetzt werden. Aus diesem
Grund wird die Migration dieser Anwendung in die Cloud untersucht.

Die durchgeführte Migration soll die Anwendung daher global verfügbar machen und vorallem den Aufwand aber auch gegebenenfalls Kosten reduzieren.
Diese Anwendung muss bisher vom Anwender aus GitHub heruntergeladen und lokal ausgeführt werden.
Zur lokalen Ausführung sind einige Anpassungen in der Konfigurationsdatei notwendig, wenn sich das ausführende System ändert.
Darüber hinaus muss die Anwendung nach jeder Aktualisierung manuell erneut heruntergeladen werden.
Das zur Verfügung stellen in der Cloud soll die Benutzung der Anwendung erleichtern und es auf lange Sicht ermöglichen, dass diese auch außerhalb der Projektumgebung verwendet werden kann.
\section{Einordnung dieser Arbeit in den wissenschaftlichen Kontext}

Grundlage für diese Arbeit bilden die zu dem Thema Cloud Migration veröffentlichten Publikationen, welche die Herausforderungen der Cloud Migration in Anbetracht der Wahl einer passenden Architektur erarbeiten. Dazu zählen unter anderem auch die Publikationen von Pahl et al., Carrasco et al. (2018) und Kazanavičius et al. (2019), die auf die Migration von \textit{legacy} Anwendungen in die Cloud eingehen und damit verbundene Herausforderungen aufzeigen. Dort wird unter anderem die Auswahl einer passenden Architektur \cite[Vgl.][S. 14]{Pahl}, der für das Unternehmen zutreffenden Migrationsstrategie \cite[Vgl.][S. 4]{Kazanavicius2019} und die Analyse potenzieller Vor- und Nachteile, von zum Beispiel der Microservice-Architektur \cite[Vgl.][S. 3]{Carrasco2018}, als Herausforderung erarbeitet.
\section{Kritische Auswahl der Forschungsmethoden}
\label{sec:auswahl_forschungsmethoden}

% Zielsetzung
% Literaturrecherche
% -hierzu zunächst aktuellen Stand der Forschung untersuchen
%  ->Thema x,y,z
% -Methodik beschreiben
% Anforderungsanalyse
% -non-functional Requirements
% -was muss beachtet werden für Cloud Anwendung
% -Anforderungen Cloud
% Use-Case Analyse
% -Werden Use-Cases immer noch erfüllt?
%  ->zufriedenstellend?
% -Architekturelle Änderungen (Client -> Web)
% Prototyping
% Evaluation

Um das Ziel dieser Arbeit zu erreichen wird zunächst der aktuelle Stand der wissenschaftlichen Forschung zu dem Thema Cloud Computing und Cloud Migration untersucht. Besonderer Fokus wird hierbei auf die Untersuchung der Herausforderungen der Cloud Migration, bevor die Durchführung eines Migrationsansatzes betrachtet wird.

Hierfür, und besonders um die aktuellen Herausforderungen zu erarbeiten, wird eine Literaturrecherche nach Döring/Bortz 2016 durchgeführt. Dazu werden einige Suchbegriffe zur systematischen Durchsuchung wissenschaftlicher Datenbanken festgelegt \cite[Vgl.][S. 158]{Doering2016}. Als Suchmaschine wurde in diesem Fall hauptsächlich Google Scholar genutzt, da hierüber eine Vielzahl verschiedener Datenbanken durchsucht wird. Darüber hinaus wurde zusätzlich noch explizit die Datenbank des Verlages Springer verwendet, aber vereinzelt auch noch die Datenbanken von arXiv, IEEE oder ACM.

Zur Recherche wurden die folgenden Begriffe verwendet: ''Cloud Migration'', ''Cloud Computing'', ''Cloud'', ''Herausforderungen Cloud Migration''

Diese werden als \textbf{''primäre Suchbegriffe''} \cite[S. 158]{Doering2016} festgelegt. Die Suche wurde darüber hinaus nach dem Schneeballsystem durchgeführt, was bedeutet, dass die Quellen der verwendeten Veröffentlichungen zur weiteren Recherche verwendet werden \cite[Vgl.][S. 160]{Doering2016}. Aus den Quellen werden außerdem weitere Begriffe als \textbf{''sekundäre Suchbegriffe''} \cite[S. 158]{Doering2016} festgelegt und zur erweiterten Suche verwendet. Zu diesen gehören unter anderem ''Migration zu PaaS'', ''Cloud Native'' und ''Refactoring''.

Zur Durchführung des Praktischen Teils dieser Arbeit wird darüber hinaus eine Anforderungsanalyse mit Use-Case Modellierung durchgeführt, um die Anforderungen an das anschließend folgende \textit{Prototyping} nach Wilde/Hess (2007) und Heinrich (2011) zu erarbeiten. Mithilfe des Prototypen soll schließlich die Cloud Migration durchgeführt und untersucht werden. Detaillierte Ausführung zur Forschung im praktischen Teil folgt in Kapitel \ref{chap:forschungsmethoden}.