\section{Literaturrecherche}
Die Literaturrecherche wird, wie vorangehend erwähnt, zur Erarbeitung der Anforderungen für die Cloud Migration eingesetzt. Damit soll herausgearbeitet werden, welche Erwartungen an die Migration der Anwendung gestellt werden und wie die zuvor erarbeiteten Vorteile des Cloud Computings umgesetzt werden sollen. Die Literaturrecherche wird, wie bereits in Kapitel \ref{sec:auswahl_forschungsmethoden} genauer erläutert nach Döring/Bortz 2016 mit ausgewählten Suchbegriffen und im Schneeballsystem durchgeführt \cite[S. 158ff]{Doering2016}.

Die \textbf{''primären Suchbegriffe''} \cite[S. 158]{Doering2016} die im Nachfolgenden für die Anforderungsanalyse verwendet werden sind ''Migration zu PaaS'', ''Cloud-Native'', ''Rebuilding'' und ''Cloud Migration''.

Die sich aus dem Schneeballsystem ergebenden \textbf{''sekundären Suchbegriffe''} \cite[S. 158]{Doering2016} sind hier unter anderem ''Skalierbarkeit'', ''CI/CD'' und ''Cloud Infrastruktur''. \pagebreak