\section{Literaturrecherche}

Die Literaturrecherche wird wie vorangehend erwähnt zur Erarbeitung von Grundlagen des Cloud Computing und
den daraus resultierenden Anforderungen für die Cloud Migration.
Diese wird, wie bereits in Kapitel \ref{sec:auswahl_forschungsmethoden} erwähnt nach Döring/Bortz 2016 durchgeführt.

Dazu werden einige Suchbegriffe zur systematischen Durchsuchung wissenschaftlicher Datenbanken festgelegt \cite[Vgl.][S. 158]{Doering2016}.
Als Suchmaschine wurde in diesem Fall hauptsächlich Google Scholar genutzt, da hierüber auch auf weitere Datenbanken
verwiesen wird. Darüber hinaus wurde vor allem noch die Datenbank des Verlages Springer verwendet, aber vereinzelt
auch noch einige weitere wie arXiv oder ACM.

Zur Recherche wurden die folgenden Begriffe verwendet: "Cloud Migration", "Cloud Computing", "Cloud", "Herausforderungen Cloud Migration"

Diese wurden als \textbf{\glqq{primäre Suchbegriffe}\grqq{}} \cite[S. 158]{Doering2016} festgelegt. Die Suche wurde darüber hinaus
nach dem Schneeballsystem durchgeführt, was bedeutet, dass die Quellen der verwendeten Veröffentlichungen zur weiteren
Recherche verwendet werden \cite[Vgl.][S. 160]{Doering2016}. Aus den Quellen wurden außerdem weitere Begriffe als
\textbf{\glqq{sekundäre Suchbegriffe}\grqq{}} \cite[S. 158]{Doering2016} festgelegt und zur erweiterten Suche verwendet.