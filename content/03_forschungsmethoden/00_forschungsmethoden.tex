\chapter{Eingesetzte Forschungsmethoden}
\label{chap:forschungsmethoden}
Ziel dieser Arbeit ist herauszuarbeiten, wie eine Anwendung in ihrer Architektur verändert werden muss, um eine Migration in die Cloud zu realisieren. Im vorangehenden Kapitel \ref{chap:grundlagen} wurden die Grundlagen des Cloud Computing und Aspekte der Cloud Migration bereits mithilfe einer Literaturrecherche erarbeitet. Im weiteren Verlauf werden auch die Anforderungen und die Vorgehensweisen zur Migration einer Anwendung in die Cloud ebenfalls mithilfe einer Literaturrecherche erarbeitet, wobei das Vorgehen für diese im folgenden Kapitel nur kurzgefasst wird, da dieses bereits in Kapitel \ref{sec:auswahl_forschungsmethoden} detailliert ausgeführt wurde.

Auf Basis der Erkenntnisse aus der Anforderungsanalyse und der anschließenden Modellierung eines Use-Cases, soll mit dem \textit{Prototyping} eine weitere Forschungsmethode eingesetzt werden, um die Vorgehensweise zur Migration einer Anwendung in die Cloud zu erarbeiten. Das Vorgehen für die Use-Case Modellierung und das Erstellen des Prototypen wird nachfolgend erläutert.

\section{Literaturrecherche}
Die Literaturrecherche wird wie vorangehend erwähnt zur Erarbeitung der Anforderungen für die Cloud Migration eingesetzt. Diese wird, wie bereits in Kapitel \ref{sec:auswahl_forschungsmethoden} genauer erläutert nach Döring/Bortz 2016 mit ausgewählten Suchbegriffen und im Schneeballsystem durchgeführt \cite[S. 158ff]{Doering2016}.

Die \glqq{primären Suchbegriffe}\grqq{} \cite[S. 158]{Doering2016} die im Nachfolgenden für die Anforderungsanalyse verwendet wurden sind \glqq{Migration zu PaaS}\grqq{}, \glqq{Refactoring}\grqq{}, \glqq{Rebuilding}\grqq{} und \glqq{Cloud Migration}\grqq{}.

Die sich aus dem Schneeballsystem ergebenen \glqq{sekundären Suchbegriffe}\grqq{} \cite[S. 158]{Doering2016} waren hier unter anderem \glqq{Skalierbarkeit}\grqq{}, \glqq{CI/CD}\grqq{} und \glqq{Cloud Infrastruktur}\grqq{}. \pagebreak
% \section{Anforderungsanalyse}

In der Anforderungsanalyse wird herausgearbeitet, welche Erwartungen es an die Anwendung gibt und wie die zuvor erarbeiteten Vorteile der Cloud Computings umgesetzt werden sollen. Dazu wird sich an den Merkmalen, wie Skalierbarkeit und Fehlertoleranz orientiert und untersucht, wie diese für die vorliegende Anwendung umgesetzt werden können oder müssen.
\section{Use-Case Modellierung und Analyse}

Die zweite in dieser Arbeit eingesetzte Methode ist eine Use-Case Modellierung mit anschließenden Tests. Das Vorgehen wurde zuvor in folgende Schritte eingeteilt:
\begin{enumerate}
    \item \textbf{Use-Case Modellierung:} Formelle Beschreibung des umzusetzenden Use-Case. Dazu werden folgende Fragen beantwortet: ''Was passiert?'', zur Beschreibung wie das Szenario starten soll, ''Was passiert danach?'', zur Beschreibung aller Schritte bis das Szenario vollendet ist (Normalverlauf) und ''Was könnte außerdem passieren?'', um zu untersuchen, was in dem Szenario möglicher Weise schiefgehen könnte (Alternativablauf) \cite[Vgl.][S. 52]{Rosenberg2007}. Dazu wird eine Tabelle als Schema eingesetzt, in welcher die einzelnen Schritte abgearbeitet werden.
    \item \textbf{Qualitätsanforderungen:} In derselben Tabelle werden zudem auch die Qualitätsanforderungen definiert, die für den jeweiligen Use-Case relevant sind. Diese müssen in der Umsetzung auf jeden Fall erfüllt werden.
    \item \textbf{Testen der Implementierung:} Testen der Implementierung durch die Definition einiger Testfälle, die sich aus den Qualitätsanforderungen ergeben und somit den Qualitätsnachweis erbringen sollen.
\end{enumerate}
\section{Prototyping}

Als weitere Forschungsmethode dieser Arbeit wird das \textit{Prototyping} eingesetzt. Ein Prototyp ist die Vorabversion einer Anwendung oder eines Systems, welche mit geringem Aufwand erzeugt werden kann. Diese wird dann in der Regel erprobt und evaluiert um neue Erkenntnisse zu gewinnen \cite[Vgl.][S. 282]{Wilde2007}\cite[Vgl.][S. 114]{Heinrich2011}.

Entgegen den Ingenieurswissenschaften meint \textit{Prototyping} jedoch nicht zwingend die letzte Entwicklungsversion vor der Fertigstellung, sondern die frühest mögliche testweise Implementierung eines Anwendungsentwurfs \cite[Vgl.][S. 114]{Heinrich2011} und muss ein System nicht zwingend vollständig abbilden \cite[Vgl.][S. 119]{Heinrich2011}.

In dem in dieser Arbeit behandelten Fall wird als Prototyp einer der Services der Anwendung umgeschrieben und in die Cloud migriert, um den damit verbundenen Aufwand untersuchen zu können, sowie Erfahrung im Umgang mit benötigten \acp{API} zu sammeln. \pagebreak

Um darüber hinaus feststellen zu können, um welche Art des Prototyping es sich handelt, werden nachfolgend die drei Umsetzungswege aufgezeigt \cite[Vgl. auch im Folgenden][S. 370]{Alpar2019}:
\begin{itemize}
    \item \textbf{Evolutionäres Prototyping} beschreibt das Entwickeln einer frühen Softwareversion und das anschließende kontinuierliche weiterentwickeln, mithilfe von Nutzerfeedback.
    \item \textbf{Exploratives Prototyping} wird zur Erarbeitung fachlicher Anforderungen eingesetzt.
    \item \textbf{Experimentelles Prototyping} dient zur Analyse der Realisierbarkeit und Demonstration der Machbarkeit eines Entwurfs durch die Analyse technischer Fragestellungen.
\end{itemize}

In der vorliegenden Arbeit wird das experimentelle Prototyping eingesetzt. Der entwickelte Prototyp soll verwendet werden, um Festzustellen ob eine Cloud Migration der ursprünglichen Anwendung ohne weiteres realisierbar ist und welche Anpassungen dafür notwendig sind.

Hierdurch sollen die Merkmale, die Vor- und Nachteile, sowie mögliche Gründe für die Migration der in dieser Arbeit untersuchten Anwendung in die Cloud abgeleitet werden. Anschließend wird versucht aus diesen konkreten Ergebnissen, allgemeine Erkenntnisse über die Migration von Anwendungen in die Cloud abzuleiten. So soll die zentrale Forschungsfrage dieser Arbeit beantwortet werden. \pagebreak