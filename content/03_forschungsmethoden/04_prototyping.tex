\section{Prototyping}

Als weitere Forschungsmethode dieser Arbeit wird das \textit{Prototyping} eingesetzt. Ein Prototyp ist die Vorabversion einer Anwendung oder eines Systems, welche mit geringem Aufwand erzeugt werden kann. Diese wird dann in der Regel erprobt und evaluiert um neue Erkenntnisse zu gewinnen \cite[Vgl.][S. 282]{Wilde2007}\cite[Vgl.][S. 114]{Heinrich2011}.

Entgegen den Ingenieurswissenschaften meint \textit{Prototyping} jedoch nicht zwingend die letzte Entwicklungsversion vor der Fertigstellung, sondern die frühest mögliche testweise Implementierung eines Anwendungsentwurfs \cite[Vgl.][S. 114]{Heinrich2011} und muss ein System nicht zwingend vollständig abbilden \cite[Vgl.][S. 119]{Heinrich2011}.

In dem in dieser Arbeit behandelten Fall wird als Prototyp einer der Services der \mbox{Anwendung} umgeschrieben und in die Cloud migriert, um den damit verbundenen Aufwand untersuchen zu können, sowie Erfahrung im Umgang mit benötigten \acp{API} zu sammeln. \pagebreak

Um darüber hinaus feststellen zu können, um welche Art des Prototyping es sich handelt, werden nachfolgend die drei Umsetzungswege aufgezeigt \cite[Vgl. auch im Folgenden][S. 370]{Alpar2019}:
\begin{itemize}
    \item \textbf{Evolutionäres Prototyping} beschreibt das Entwickeln einer frühen Softwareversion und das anschließende kontinuierliche weiterentwickeln, mithilfe von Nutzerfeedback.
    \item \textbf{Exploratives Prototyping} wird zur Erarbeitung fachlicher Anforderungen eingesetzt.
    \item \textbf{Experimentelles Prototyping} dient zur Analyse der Realisierbarkeit und Demonstration der Machbarkeit eines Entwurfs durch die Analyse technischer Fragestellungen.
\end{itemize}

In der vorliegenden Arbeit wird das experimentelle Prototyping eingesetzt. Der entwickelte Prototyp soll verwendet werden um festzustellen, ob eine Cloud Migration der ursprünglichen Anwendung ohne weiteres realisierbar ist und welche Anpassungen dafür notwendig sind.

Hierdurch sollen die Merkmale, die Vor- und Nachteile, sowie mögliche Gründe für die Migration der in dieser Arbeit untersuchten Anwendung in die Cloud abgeleitet werden. Anschließend wird versucht aus diesen konkreten Ergebnissen, allgemeine Erkenntnisse über die Migration von Anwendungen in die Cloud abzuleiten. So soll die zentrale Forschungsfrage dieser Arbeit beantwortet werden. \pagebreak