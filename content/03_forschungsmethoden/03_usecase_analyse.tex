\section{Use-Case Modellierung und Analyse}

Die dritte in dieser Arbeit eingesetzte Methode ist eine Use-Case Modellierung mit anschließenden Tests. Das Vorgehen wurde zuvor in folgende Schritte eingeteilt:
\begin{enumerate}
    \item \textbf{Use-Case Modellierung:} Formelle Beschreibung des umzusetzenden Use-Case, durch Beantwortung der Fragen \glqq{Was passiert?}\grqq{}, zur Beschreibung wie das Szenario starten soll, \glqq{Was passiert danach?}\grqq{}, zur Beschreibung aller Schritte bis das Szenario vollendet ist und \glqq{Was könnte außerdem passieren?}\grqq{}, um zu untersuchen, was in dem Szenario möglicher Weise schiefgehen könnte \cite[Vgl.][S. 52]{Rosenberg2007}.
    \item \textbf{Akzeptanzkriterien:} Festlegen der Kriterien, welche dazu führen die Analyse als erfolgreich zu bewerten.
    \item \textbf{Testen der Implementierung:} Testen der Implementierung und Analyse des Use-Cases und seiner Subtasks.
\end{enumerate}