\section{Use-Case Modellierung und Analyse}

Die zweite in dieser Arbeit eingesetzte Methode ist eine Use-Case Modellierung mit anschließenden Tests. Das Vorgehen wurde zuvor in folgende Schritte eingeteilt:
\begin{enumerate}
    \item \textbf{Use-Case Modellierung:} Formelle Beschreibung des umzusetzenden Use-Case. Dazu werden folgende Fragen beantwortet: ''Was passiert?'', zur Beschreibung wie das Szenario starten soll, ''Was passiert danach?'', zur Beschreibung aller Schritte bis das Szenario vollendet ist (Normalverlauf) und ''Was könnte außerdem passieren?'', um zu untersuchen, was in dem Szenario möglicher Weise schiefgehen könnte (Alternativablauf) \cite[Vgl.][S. 52]{Rosenberg2007}. Dazu wird eine Tabelle als Schema eingesetzt, in welcher die einzelnen Schritte abgearbeitet werden.
    \item \textbf{Qualitätsanforderungen:} In derselben Tabelle werden zudem auch die Qualitätsanforderungen definiert, die für den jeweiligen Use-Case relevant sind. Diese müssen in der Umsetzung auf jeden Fall erfüllt werden.
    \item \textbf{Testen der Implementierung:} Testen der Implementierung durch die Definition einiger Testfälle, die sich aus den Qualitätsanforderungen ergeben und somit den Qualitätsnachweis erbringen sollen.
\end{enumerate}