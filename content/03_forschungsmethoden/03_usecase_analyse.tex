\section{Use-Case Modellierung und Analyse}

Die zweite in dieser Arbeit eingesetzte Methode ist eine Use-Case Modellierung mit anschließenden Tests. Das Vorgehen wurde zuvor in folgende Schritte eingeteilt:
\begin{enumerate}
    \item \textbf{Use-Case Modellierung:} Formelle Beschreibung des umzusetzenden Use-Case. Dazu werden folgende Fragen beantwortet: ''Was passiert?'', zur Beschreibung wie das Szenario starten soll, ''Was passiert danach?'', zur Beschreibung aller Schritte bis das Szenario vollendet ist (Normalverlauf) und ''Was könnte außerdem passieren?'', um zu untersuchen, welche Fehler in dem Szenario möglicherweise auftreten können (Alternativablauf) \cite[Vgl.][S. 52]{Rosenberg2007}. Dazu wird eine Tabelle als Schema eingesetzt, in welcher die einzelnen Schritte aufgeführt werden.
    \item \textbf{Qualitätsanforderungen:} In derselben Tabelle werden zudem auch die Qualitätsanforderungen definiert, die für den jeweiligen Use-Case relevant sind. Diese müssen in der Umsetzung definitiv erfüllt werden.
    \item \textbf{Testen der Implementierung:} Zudem werden Testfälle definiert, durch welche die Implementierung aud die Erfüllung ihrer funktionalen Anforderungen hin untersucht werden soll. Die Erfüllung dieser Anforderungen dient dazu, den Qualitätsnachweis zu erbringen.
\end{enumerate}