%!TEX root = ../../dokumentation.tex

\chapter{Theoretische Grundlagen}
\label{chap:grundlagen}

Das folgende Kapitel bietet einen Überblick über den aktuellen Stand der Forschung und aktuelle Entwicklungen im Themenbereich des Cloud Computing und im Speziellen der Cloud Migration.

\section{Cloud Computing}

Im folgenden Unterkapitel werden die Grundlagen und eine Definition des Cloud Computing erarbeitet. Hierbei werden die Grundlegenden Konzepte,
Bereitstellungsmodelle und Abstraktionsebenen des Cloud Computing erläutert.

\subsection{Was ist Cloud Computing}

Nach dem \ac{NIST} ist Cloud Computing ein Modell der Zurverfügungstellung von Computing Ressourcen
(z.B. Netzwerke, Server, Speicher, Anwendungen und Services),die über das Netzwerk erreichbar sind und mit geringem Managementaufwand
schnell freigegeben und bereitgestellt werden können \cite[Vgl.][S. 2]{Mell2011}.

\begin{wrapfigure}{r}{0.45\textwidth}
\centering
\includegraphics[height=0.3\textwidth]{xaas_maenhaut.png}
\caption{Eine Übersicht der Cloud Service Modelle \cite[S. 33]{Maenhaut2016}}
\label{fig:XaaS}
\end{wrapfigure}

Nach Hentschel und Leyh (2018), Zhao (2014), Maenhaut (2016) und Surianarayanan (2019) kann man Cloud Services grundsätzlich in drei Abstraktionsebenen einteilen. Diese sind \textbf{\ac{SaaS}},
\textbf{\ac{PaaS}} und \textbf{\ac{IaaS}}, welche auch zu \ac{XaaS} zusammengefasst werden
\cite[Vgl.][S. 9]{Reinheimer2018}\cite[Vgl.][S. 143f]{Zhao2014}\cite[Vgl.][S. 32ff]{Maenhaut2016} und \cite[Vgl.][S. 226ff]{Surianarayanan2019}.

Die unterste der drei genannten Abstraktionsschichten ist die \ac{IaaS}, welche die Basisinfrastruktur, wie zum Beispiel Netzwerk, Server oder Speicher,
bereitstellt. Diese Infrastruktur kann sowohl physisch als auch virtuell zur Verfügung gestellt werden \cite[Vgl.][S. 9f]{Reinheimer2018}.
Die darüberliegende Schicht ist die \ac{PaaS}, welche auf der Infrastruktur zusätzlich noch eine Basis zur Anwendungsentwicklung bietet, indem zum Beispiel
bereits ein Betriebssystem und eine Datenbank installiert sind oder eine andere Programmierumgebung verwendet werden kann \cite[Vgl.][S. 10]{Reinheimer2018}.
Die darüberliegende Schicht ist die \ac{SaaS}, welche standardisierte Anwendungen zur Verfügung stellt und sich somit direkt an den Endnutzer richtet
\cite[Vgl.][S. 11]{Reinheimer2018}.

In Abbildung \ref{fig:XaaS} wird eine Übersicht der drei Abstraktionsebenen angezeigt und dargestellt, wie es sich mit der Flexibilität über diese verhält \cite[Vgl.][S. 33]{Maenhaut2016}.

Generell wird die Cloud darüber hinaus in drei Organisationsdimensionen eingeteilt \cite[Vgl. auch im Folgenden][S. 7ff]{Reinheimer2018}:
\begin{itemize}
\item \textbf{Private Cloud:} Die Private Cloud bietet die exklusive Nutzung durch eine Organisation der darunterliegenden Infrastruktur. 
Die IT-Infrastruktur einer Private Cloud kann entweder im Unternehmenseigenen Rechenzentrum untergebracht oder auch
von Dienstleistern bereitgestellt werden.
\item \textbf{Public Cloud:} In der Public Cloud ist die Infrastruktur für mehr Anwender zugänglich und muss geteilt werden. Dafür muss als Anwender oft
auch nur für die tatsächlich genutzte Leistung gezahlt werden. Da die Infrastutktur jedoch gleichzeitig von vielen genutzt wird, ist zum Beispiel der
Betrieb von sicherheitskritischen Anwendungen schwierig.
\item \textbf{Hybrid Cloud:} Die Hybrid Cloud bildet eine kombinierte Anwendung aus der Public Cloud und der Private Cloud. Diese bietet dem Anwender die
Möglichkeit gewisse Anwendungen in die Public Cloud auszulagern, ohne die Vorteile der Private Cloud für sicherheitsrelevante Anwendungen aufgeben zu msüssen.
Darüber hinaus kann bei einem Hybrid Cloud modell die Rechenleistung der Private Cloud bei Spitzenlast durch die Public Cloud erweitert werden. 
\end{itemize}

\pagebreak

\subsection{Entwicklung des Cloud Computing}

Die Entwicklung des Cloud Computing und dessen Vorgängerkonzepte is bis in die 90er Jahre zurückzuführen.
Ein von Hentschel und Leyh (2018) hervorgehobender Vorgänger ist das sogrnannte Grid Computing.
Damit war bereits eine dezentrale Ressourcenkontrolle mir standardisierten Protokollen und
Schnittstellen realisiert. Das Cloud Computing bietet vergleichbare Eigenschaften, jedoch rückt der
Fokus hier auf wirtschaftliche Kriterien und die Zenralisierung von Ressourcen zum Beispiel in
Rechenzentren \cite[Vgl.][S. 5f]{Reinheimer2018}.

Salesforces ear eines der ersten Unternehmen, welches 1999 Anwendungen über eine Webseite bereitgestellt hat,
gefolgt von den \ac{AWS} in 2002, welche Speicher und Rechenleistung als Services bereitstellten \cite[Vgl.][S. 17f]{Srivastava2018}.

\pagebreak

\subsection{Herausforderungen -> Migration von Legacy Anwendungen}

Aus der vorangehend erläuterten Entwicklung des Cloud Computing ergeben sich verschiedene Herausforderungen.
Darunter fällt unter anderem die Migration von Legacy Anwendungen zu Cloud Anwendungen, welche gegeben durch die verschiedenen Service Level
auf unterschiedlichen Wegen realisiert werden kann.
\section{Cloud Native Anwendungen}

% Was ist eine Cloud Native Anwendung?
% Merkmale

Von \textit{Cloud-native} Anwendungen wird erwartet, dass diese schnell hoch skalieren können um einer steigenden Nutzerzahl gerecht zu werden \cite[Vgl.][S. 1ff]{Armbrust2009} \cite[Vgl.][S. 234]{Villamizar2017}. Um die Vorteile der Cloud aber auch hinsichtlich der Kosten nutzen zu können müssen die Services genauso herunter zu skalieren sein \cite[Vgl.][S. 884]{Adzic2017} \textbf{TODO} \pagebreak
\section{Serverless Computing} % vs. Server-Aware
\label{sec:serverless-serveraware}

% \subsection{Server-Aware}
% \label{sec:server-aware}

% Der englische Begriff \textit{Server Aware} bedeutet auf deutsch, sich dem Server Bewusst sein. Bei der Nutzung von \ac{IaaS} hat der Entwickler die meiste Kontrolle über Anwendungen und die Infrastruktur in der Cloud und ist verantwortlich für die Bereitstellung von zum Beispiel Hardwareressourcen und \acp{VM} \cite[Vgl.][S. 3]{Baldini2017}.

% \subsection{Serverless}
% \label{sec:serverless}

% Beim Einsatz von \ac{PaaS} und \ac{SaaS} Service Modellen ist der Entwickler ''nichtwissend'' (\textit{unaware}) über die Cloud Infrastruktur \cite[Vgl.][S. 3]{Baldini2017}. \textit{Serverless}, also Serverlos ist eigentlich kein zutreffender Name, da die Infrastruktur und Server nach wie vor existieren, der Entwickler sich lediglich nicht darum kümmern muss, wie diese aussehen \cite[Vgl.][S. 5]{Baldini2017}.

In den vergangenen Jahren hat sich \textit{Serverless Computing} als neues Paradigma des Cloud Computing entwickelt \cite[Vgl.][S. 44]{Castro2019}\cite[Vgl.][S. 64]{Anel2020}. 

\textit{Serverless} Computing bedeutet, dass die Entwickler unabhängig von der Cloud-Infrastruktur entwickeln können. Der Begriff wird noch besser durch den gegenteiligen Begriff \textit{Server Aware} (dt. Server bewusst) verständlich, da dieser deutlich macht, dass Entwickler sich dabei der Server und Infrastruktur bewusst sein müssen und dieses Bewusstsein bei \textit{Serverless} nicht notwendig ist \cite[Vgl.][S. 5]{Jonas2019}\cite[Vgl.][S. 1]{Hellerstein2018}\cite[Vgl.][S. 46]{Castro2019}\cite[Vgl.][S. 64]{Anel2020}.

Darüber hinaus bringt \textit{Serverless} den Vorteil mit sich, dass eine Abrechnung der Kosten nur dann erfolgt, wenn die Funktionen auch ausgeführt werden, da die Ressourcen nur für den Ausführungszeitraum bereitgestellt werden \cite[Vgl.][S. 46]{Castro2019}. Mit \textit{Serverless} kann die Infrastruktur effizienter genutzt werden, da diese nicht dauerhaft für bestimmte Anwendungen reserviert ist und die Produktivität in der Entwicklung steigt \cite[Vgl.][S. 9]{Jonas2019}.
\pagebreak
\section{Herausforderungen der Cloud Migration}

% Merkmale von Cloud nativen Anwendungen in legacy Applikationen umsetzen
% Verlass auf Box-API muss gegeben sein
% https://developers.redhat.com/articles/2021/06/14/application-modernization-patterns-apache-kafka-debezium-and-kubernetes#application_modernization_in_context

Cloud Computing birgt jedoch auch einige Herausforderungen, welche in der Entwicklung und Migration komplexe Aufgaben mit sich bringen können.

Neben der steigenden Nutzung von Cloud Computing und der Entwicklung Cloud basierter Anwendungen können auch legacy Anwendungen von Cloud Computing profitieren, woraus sich der Trend abzeichnet, dass auch solche Anwendungen auf eine Cloud Infrastruktur migriert werden, um die Ressourcen dieser nutzen zu können und Kosten zu sparen \cite[Vgl.][S. 31]{Maenhaut2016}.

Abhängig vom gewählten Migrationsansatz sind Änderungen in der Anwendung vorzunehmen, damit die Vorteile der Cloud genutzt werden können. Verbesserung des Anwendungsdesigns und die Optimierung von Ressourcennutzung sind nach Feathers (2004) zwei der vier möglichen Hauptgründe Änderungen an Software vorzunehmen \cite[Vgl.][S. 3]{Feathers2004}. Diese lassen sich auch auf die vorzunehmenden Änderungen für die Cloud Migration übertragen. Die Herausforderung die bei der Änderung von Software aufkommt, ist sicherzustellen, dass das grundlegende Verhalten der Anwendungen nicht verändert wird. Die Schwierigkeit liegt darin, oft nicht genau zu wissen wie Stark das Verhalten verändert wird, wenn Änderungen vorgenommen werden \cite[Vgl.][S. 7]{Feathers2004}. \pagebreak
\section{Migrationsansätze}
\label{sec:migrationsansaetze}
% Architekturpattern
% Wann Einsatz der jeweiligen Pattern?

Im folgenden Kapitel wird auf die unterschiedlichen Migrationsmethoden des Cloud Computing eingegangen. Durch die unterschiedlichen Abstraktionsebenen bedingt gibt es verschiedene Ansätze, wie die Cloud Migration realisierbar ist \cite[Vgl.][S. 226]{Surianarayanan2019}, angefangen mit dem auch als ''Lift and Shift'' bezeichneten Rehosting, bis hin zur Migration zu \ac{SaaS} verbunden mit der Entwicklung Cloud nativer Anwendungen \cite[Vgl.][S. 144]{Zhao2014}.

\subsection{Rehosting}
Nach Zhao (2014) wird das \textit{Rehosting} vorallem für die Migration zu \ac{IaaS} eingesetzt \cite[Vgl.][S. 144]{Zhao2014}. Umgangssprachlich wird diese Vorgehensweise auch als ''Lift and Shift'' bezeichnet \cite[Vgl.][]{NetApp}. Bei diesen Strategien werden Anwendungen lediglich auf einer anderen Hardwareplattform installiert, die Anwendungsarchitektur bleibt dabei unverändert. Dieser Ansatz bietet eine schnelle Lösung zur Migration \cite[Vgl.][]{CIO}.

\textbf{(Siehe Feedback Steffen: Ein/zwei Sätze zu ''Lift'' ausführen)}

Der wahrscheinlich größte Vorteil der Migration zu \ac{IaaS} ist, dass die Anwendungsarchitektur nicht verändert werden muss, weshalb das \textit{Rehosting} vorallem dann eingesetzt wird, wenn die Migration schnell und ohne großen Aufwand vollzogen werden soll. Ein Nachteil ist dagegen, dass bei der Nutzung von \ac{IaaS} nicht die vollen Möglichkeiten der Cloud ausgeschöpft werden können, da die Anwendungen nicht entsprechend angepasst werden \cite[Vgl.][]{CIO}.

\subsection{Repurchase}
\textit{Repurchase} bedeutet, Software in \ac{SaaS} Modellen einzukaufen \cite[Vgl.][S. 2]{Ahmad2018}. Die Migration zu einem \ac{SaaS}-Modell ist nach Zhao (2014) neben dem Ersetzen einer Software durch eingekaufte \acs{SaaS} auch durch entsprechendes überarbeiten von Software nach \ac{SaaS}-Prinzipien oder gänzlicher Umstrukturierung dieser \cite[Vgl.][S. 144]{Zhao2014}. 

Das Aufgeben einer Anwendung und vollständige Ersetzen dieser durch Nutzung des \ac{SaaS} Modells bedarf eines größeren Aufwand und ein Entwicklungsteam zur Umsetzung der Anforderungen \cite[Vgl.][]{CIO}. \pagebreak

\subsection{Refactoring/Rebuilding}
Das \textit{Refactoring} oder \textit{Rebuilding} von Anwendungen gehört nach Zhao (2014) zu den eingesetzten Strategien für die Migration zu \ac{PaaS} \cite[Vgl.][S. 144]{Zhao2014}. \textit{Refactoring} bedeutet, die Architektur der bisherigen Anwendung anzupassen, um Cloud-native Features \cite[Vgl.][S. 2]{Ahmad2018} und somit die Vorteile der Cloud Plattform nutzen zu können \cite[Vgl.][]{CIO}. Darunter fällt zum Beispiel das Umsetzen entsprechender Service Topologien im Code \cite[Vgl.][S. 2]{Holmes2018}.

Durch die vom Cloud Provider teils vorgegebene Plattform (z.B. Middleware und Datenbanken) müssen Anwendungen dieser entsprechend angepasst werden \cite[Vgl.][S. 227]{Surianarayanan2019}. Der Mehrwert einer Migration zu \ac{PaaS} ist, dass der Anwender die IT-Infrastruktur nicht mehr verwalten muss und somit Aufwände reduziert werden können \cite[Vgl.][S. 6]{Pahl}.

Auch \textit{Rebuilding} wird eingesetzt um die Vorteile der Cloud mit Cloud-native Features zu nutzen, jedoch wie vom Namen schon abzuleiten durch das neu ''bauen'' einer Anwendung statt dem anpassen \cite[Vgl.][S. 2]{Ahmad2018}.

% Strangler Pattern
\subsection{Strangler Pattern}
Eine Strategie die Migration einer Anwendung umzusetzen ist das von Martin Fowler eingeführte \textit{Strangler Pattern}. Den Namen hat Fowler von der im Englischen \textit{Strangler Fig} genannten Würgefeige, die von den Ästen eines Baumes herab wächst, bis die diesen langsam abwürgt und ersetzt und metaphorisch für den Prozess des Neuschreibens einer Anwendung übernommen \cite[Vgl.][]{Fowler2004}\cite[Vgl.][]{Ibryam2021}.

Zur Anwendung des \textit{Strangler Pattern} werden folgende Schritte durchgeführt \cite[Vgl. auch im Folgenden][]{Ibryam2021}:

\begin{enumerate}
    \item Identifizieren funktionaler Grenzen zur Abgrenzung einzelner Services voneinander
    \item Funktionalität einzelner Services migrieren
    \item Migration / Anpassung der Datenbank
    \item Testweiser Release der neuen Services
\end{enumerate}

Die neue Anwendung wird so parallel zu der noch aktiven, nicht migrierten Anwendung aufgebaut und die Services nach und nach in Betrieb genommen, bis die alte Anwendung vollständig ersetzt wurde und nicht mehr benötigt wird \cite[Vgl.][]{MarshBoourdon2019}\cite[Vgl.][]{Ibryam2021}\cite[Vgl.][]{Fowler2004}.
\pagebreak

% \subsection{''Big Bang'' Migration}
% \dots
% \pagebreak
\section{Architekturstile}

\subsection{Monolithische Anwendungen}
% Was sind monolithishue Anwendungen?
% In Bezug auf Cloud Native -> warum nicht geeignet

\subsection{Microservice Architektur}
% Wann/Wie

\textbf{TODO}

%title wird unter dem Bsp. abgedruckt
%caption wird im Verzeichnis abgedruckt
%label wird zum referenzieren benutzt, muss einzigartig sein.

% \begin{lstlisting}[caption=Code-Beispiel, label=Bsp.1]
% public class HelloWorld {
% 	public static void main (String[] args) {
% 		// Ausgabe Hello World!
% 		System.out.println("Hello World!");
% 	}
% }
% \end{lstlisting}

% %language ändert die Sprache. (Wenn nur eine Sprache verwendet wird, kann diese Sprache in einstellungen.tex geändert werden. Standardmäßig Java.)
% \begin{lstlisting}[caption=Python-Code, label=Python-Code, title=Titel des Python-Codes,language=Python]
% def quicksort(liste):
% if len(liste) <= 1:
% 	return liste
% pivotelement = liste.pop()
% links = [element for element in liste if element < pivotelement]
% rechts = [element for element in liste if element >= pivotelement]
% return quicksort(links) + [pivotelement] + quicksort(rechts)
% # Quelle: http://de.wikipedia.org/wiki/Python_(Programmiersprache)
% \end{lstlisting}

% \section{Verweis auf Code}
% Verweis auf den Code \autoref{Bsp.1}.\\
% und der Python-Code \autoref{Python-Code}.

% Zweite Erwähnung einer Abkürzung \ac{AGPL} (Erlärung wird nicht mehr angezeigt)