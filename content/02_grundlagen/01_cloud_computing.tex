\section{Cloud Computing}

Im folgenden Unterkapitel werden die Grundlagen und eine Definition des Cloud Computing erarbeitet um den Kontext dieser Arbeit herleiten zu können.
Hierbei wird vorallem auf die Definition des Cloud Computing nach dem NIST und einiger Grundlagen nach Reinheimer2018 eingegangen.

\subsection{Was ist Cloud Computing}

Cloud Computing ist ein Modell der Zurverfügungstellung eines universell erreichbaren,
günstigen Netzwerkzugangs zu einer Ansammlung konfigurierbarer Computing Ressourcen
(z.B. Netzwerke, Server, Speicher, Anwendungen und Services), die mit minimalem Managementaufwand schnell freigegeben und bereitgestellt werden können
\cite[Vgl.][S. 2]{Mell2011}.

Nach Hentschel und Leyh 2018 kann man Cloud Services grundsätzlich in drei Abstraktionsebenen einteilen. Diese sind \acf{SaaS},
\acf{PaaS} und \acf{IaaS}, welche auch zu \acf{XaaS} zusammengefasst werden
\cite[Vgl.][S. 9]{Reinheimer2018}.

Die unterste der drei genannten Abstraktionsschichten ist die \acf{IaaS}, welche die Basisinfrastruktur, wie zum Beispiel Netzwerk, Server oder Speicher,
bereitstellt. Diese Infrastruktur kann sowohl physisch als auch virtuell zur Verfügung gestellt werden \cite[Vgl.][S. 9f]{Reinheimer2018}.
Die darüberliegende Schicht ist die \acf{PaaS}, welche auf der Infrastruktur zusätzlich noch eine Basis zur Anwendungsentwicklung bietet, indem zum Beispiel
bereits ein Betriebssystem und eine Datenbank installiert sind oder eine andere Programmierumgebung verwendet werden kann \cite[Vgl.][S. 10]{Reinheimer2018}.
Die darüberliegende Schicht ist die \acf{SaaS}, welche standardisierte Anwendungen zur Verfügung stellt und sich somit direkt an den Endnutzer richtet
\cite[Vgl.][S. 11]{Reinheimer2018}.

\subsection{Entwicklung des Cloud Computing}

\subsection{Herausforderungen -> Migration von Legacy Anwendungen}