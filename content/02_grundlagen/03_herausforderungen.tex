\section{Herausforderungen}

Cloud Computing birgt jedoch auch einige Herausforderungen, welche in der Entwicklung und Migration komplexe Aufgaben mit sich bringen können.

Neben der steigenden Nutzung von Cloud Computing und der Entwicklung Cloud basierter Anwendungen können auch legacy Anwendungen von Cloud Computing profitieren, woraus sich der trend abzeichnet, dass auch solche Anwendungen auf eine Cloud Infrastruktur migriert werden um die Ressourcen dieser nutzen zu können und Kosten zu sparen \cite[Vgl.][S. 31]{Maenhaut2016}.

Abhängig vom gewählten Migrationsansatz sind Änderungen in der Anwendung vorzunehmen, damit die Vorteile der Cloud genutzt werden können. Verbesserung des Anwendungsdesigns und die Optimierung von Ressourcennutzung sind nach Feathers (2004) zwei der vier möglichen Hauptgründe Änderungen an Software vorzunehmen \cite[Vgl.][S. 3]{Feathers2004}. Diese lassen sich auch auf die vorzunehmenden Änderungen für die Cloud Migration übertragen. Die Herausforderung die bei der Änderung von Software aufkommt, ist sicherzustellen, dass das grundlegende Verhalten der Anwendungen nicht verändert wird. Die Schwierigkeit liegt darin, oft nicht genau zu wissen wie Stark das Verhalten verändert wird, wenn Änderungen vorgenommen werden \cite[Vgl.][S. 7]{Feathers2004}.