\section{Herausforderungen der Cloud Migration}
\label{sec:herausforderungen}

% Merkmale von Cloud nativen Anwendungen in legacy Applikationen umsetzen
% Verlass auf Box-API muss gegeben sein
% https://developers.redhat.com/articles/2021/06/14/application-modernization-patterns-apache-kafka-debezium-and-kubernetes#application_modernization_in_context

Cloud Computing birgt jedoch auch einige Herausforderungen, welche in der Entwicklung und Migration komplexe Aufgaben mit sich bringen können.

\subsection{Vorteile der Cloud nutzen}

Neben der steigenden Nutzung von Cloud Computing und der Entwicklung Cloud-basierter Anwendungen können auch \textit{legacy} Anwendungen von Cloud Computing profitieren. Daraus zeichnet sich der Trend ab, dass auch solche Anwendungen auf eine Cloud Infrastruktur migriert werden. Diese Entwicklung ermöglicht es, die Ressourcen der Cloud nutzen zu können und Kosten zu sparen \cite[Vgl.][S. 31]{Maenhaut2016}.

Abhängig vom gewählten Migrationsansatz sind Anpassungen in der Anwendung vorzunehmen, meist um die Vorteile der Cloud ausnutzen zu können. Verbesserung des Anwendungsdesigns und die Optimierung von Ressourcennutzung sind nach Feathers (2004) zwei der vier möglichen Hauptgründe Anpassungen an Software vorzunehmen \cite[Vgl.][S. 3]{Feathers2004}. Diese lassen sich auch auf die vorzunehmenden Anpassungen für eine Cloud Migration übertragen. Eine Herausforderung, die bei der Anpassung von Software aufkommt, ist es sicherzustellen das grundlegende Verhalten der Anwendungen nicht zu beeinträchtigen. Die Schwierigkeit liegt darin, oft nicht genau erkennen zu können, wie Stark das Verhalten der Anwendung auf diese Änderungen reagiert \cite[Vgl.][S. 7]{Feathers2004}. 

\subsection{Entwicklung von Cloud-basierten Anwendungen}

Eine weitere Herausforderung in der Anwendungsentwicklung ist, die Anwendung und den Entwicklungsprozess so zu gestalten, dass mehrere Entwickler gleichzeitig an verschiedenen Funktionen der Anwendung arbeiten können, ohne sich dabei gegenseitig in die Quere zu kommen. Auch darauf ist entsprechend bei der Cloud Migration von \textit{legacy} Anwendungen zu achten \cite[Vgl.][]{Ibryam2021}.

Außerdem wird im Zuge der Migration meist erwartet, dass die Vorteile der Cloud, wie zum Beispiel eine effiziente Skalierung der Anwendung, umgesetzt werden, was zusätzlichen Aufwand zur reinen Migration bedeutet \cite[Vgl.][]{Ibryam2021}. \pagebreak

\subsection{Datenschutz und Datensicherheit in der Cloud}
Personenbezogene und sensible (z. B. geschäftliche) Daten bedürfen einer guten Datensicherheit, vor allem im Kontext des Cloud Computing und dort insbesondere für die Public Cloud, wo Speicherressourcen nicht direkt in den Händen des Anwenders liegen, sondern in den Rechenzentren der Provider verarbeitet und gespeichert werden \cite[Vgl.][S. 1ff]{Sun2019}. Aus diesem Grund ist es wichtig, mögliche Risiken sowie die damit verbundenen Herausforderungen zu identifizieren \cite[Vgl.][S. 3]{Sun2019}.

Folgende Risiken können für Datenschutz und Datensicherheit in der Cloud existieren \cite[Vgl. auch im folgenden][S. 694]{Kumar2018}:

\begin{itemize}
    \item Risiken in Zusammenhang mit \textit{\ac{CIA}}\footnote{dt. Konsistenz, Integrität und Verfügbarkeit, }
    \item Herausforderungen bei der Authentifizierungs- und Zugriffskontrolle
    \item Fehlerhafte Authentifizierungs-, Sitzungs- und Zugriffskontrollen
    \item Weitere Risiken, die durch z. B. Speicherort der Daten, Mehrbenutzerfähigkeit oder Backups entstehen können
\end{itemize}

\pagebreak