\section{Architekturstile im Hintergrund des Cloud Computing}
\label{sec:architekturstile}
Die Architektur von Anwendungen ist in den meisten Fällen entweder monolithisch oder als Microservice Architektur umgesetzt \cite[Vgl.][S. 150]{Gos2020}. Das nachfolgende Kapitel beschreibt diese Architekturstile mit ihren Eigenschaften und aus welchem Grund die Microservice Architektur für den Betrieb in der Cloud bevorzugt eingesetzt wird \cite[Vgl.][S. 1]{Villamizar2015}.

\subsection{Monolithische Anwendungsarchitektur}
% Was sind monolithische Anwendungen?
Monolithische Anwendungen, wie sie in der Vergangenheit oft eingesetzt wurden, bestehen aus verschiedenen Komponenten, die zu einem Programm kombiniert werden. Die einzelnen Komponenten funktionieren nur zusammen innerhalb des Monolithen, einzelne Komponenten können also nicht unabhängig betrieben werden. \cite[Vgl.][S. 1]{Gos2020}.

\begin{figure}[H]
    \centering
    \includegraphics[width=0.65\textwidth]{monolith.png}
    \caption{Beispiel einer monolithischen Architektur \cite[Nachbildung angelehnt an][S. 150]{Gos2020}}
    \label{fig:monolith}
\end{figure}

Abbildung \ref{fig:monolith} zeigt, wie eine monolithische Architektur aufgebaut sein kann. Hier sind die Komponenten eng miteinander verwoben und voneinander abhängig.
% In Bezug auf Cloud Native -> warum nicht geeignet
\pagebreak

\subsection{Microservice Architektur}
Der Einsatz von Microservice Architekturen hat in den letzten Jahren stark zugenommen \cite[Vgl.][S. 150]{Gos2020}. Microservices bedeutet, dass eine Anwendung aus einer Zusammenstellung einzelner Services besteht, wobei jeder Service einen Teil der Business-Logik erfüllt. Diese Services sind dabei, wie in Abbildung \ref{fig:microservice} dargestellt voneinander unabhängig um keinen Single-Point of Failure zu erzeugen \cite[Vgl.][S. 150]{Gos2020}\cite[Vgl.][]{Janssen2021}\cite[Vgl.][]{Fowler2014}.

\begin{figure}[H]
    \centering
    \includegraphics[width=0.65\textwidth]{microservice.png}
    \caption{Beispiel einer Microservice Architektur \cite[Nachbildung angelehnt an][S. 150]{Gos2020}}
    \label{fig:microservice}
\end{figure}

In Bezug auf Cloud-native Anwendungen bringen Microservices eine effizientere Skalierbarkeit mit sich, da jeder Service einzeln je nach Bedarf skaliert werden kann und nicht die komplette monolithische Anwendung skaliert werden muss \cite[Vgl.][]{Janssen2021}.

Die Entwicklung von Microservices hat außerdem den Vorteil, dass der Blick im Entwicklungsprozess separiert auf die einzelnen Services gelegt werden kann und der Entwickler somit nicht jederzeit den ganzen Monolithen im Auge haben muss, um zu verhindern, dass einzelne Funktionen sich gegenseitig beeinflussen \cite[Vgl.][]{Janssen2021}. \pagebreak

% In Bezug auf Cloud Native -> weil skalierbar (kleine Einheiten etc...)
% Fehlertoleranter
% Asynchrones Messaging (-> keine Fails, wenn Service ausfällt)