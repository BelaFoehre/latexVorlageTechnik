\section{Migrationsansätze}

Im folgenden Kapitel wird auf die unterschiedlichen Migrationsmethoden des Cloud Computing eingegangen.
Durch die unterschiedlichen Servicelevel bedingt gibt es verschiedene Ansätze, wie die Cloud Migration realisierbar ist \cite[Vgl.][S. 226]{Surianarayanan2019},
angefangen mit dem sogenannten \glqq{Lift and Shift}\grqq{}, bis hin zur Entwicklung Cloud nativer Anwendungen \cite[Vgl.][S. 144]{Zhao2014}.

\subsection{Migration zu IaaS (Replatforming/Rehosting)}
Nach Zhao (2014) ist das Replatforming oder Rehosting die vorgeschlagene Strategie für die Migration zu IaaS \cite[Vgl.][S. 144]{Zhao2014}.
Umgangssprachlich wird diese Vorgehensweise auch als \glqq{Lift and Shift}\grqq{} bezeichnet \cite[Vgl.][]{NetApp}.
Bei diesen Strategien werden Anwendungen lediglich auf einer anderen Hardwareplattform installiert, die Anwendungsarchitektur bleibt dabei
unverändert. Dieser Ansatz bietet eine schnelle Lösung zur Migration \cite[Vgl.][]{CIO}.

Der wahrscheinlich größte Vorteil der Migration zu IaaS ist, dass die Anwendungsarchitektur nicht verändert werden muss und die Migration somit
schnell und ohne großen Aufwand vollzogen werden kann. Ein Nachteil ist dagegen, dass die Migration zu IaaS nicht die vollen Möglichkeiten der
Cloud ausnutzt.

\subsection{Migration zu PaaS (Refactoring/Rebuilding)}
Das Refactoring oder Rebuilding von Anwendungen gehört nach Zhao (2014) zu den empfohlenen Strategien für die Migration zu PaaS \cite[Vgl.][S. 144]{Zhao2014}.
Rebuilding bedeutet, den bisherigen Code zu re-architekten um die Vorteile der Cloud Plattform nutzen zu können \cite[Vgl.][]{CIO}.

\subsection{Migration zu SaaS (Cloud Native Entwicklung)}
Der Schritt zu SaaS ist nach Zhao (2014) auf verschiedenen Wegen zu erreichen. Demnach sei eine Anwendung durch einen SaaS Service entweder zu ersetzen,
enstprechend der SaaS Prinzipien zu überarbeiten oder zu einem SaaS Service unzustrukturieren \cite[Vgl.][S. 144]{Zhao2014}.

Das Augeben einer Anwendung und vollständige Ersetzen dieser durch Nutzung des SaaS Modells bedarf eines
größeren Aufwand und ein Entwicklungsteam zur Umsetzung der Anforderungen \cite[Vgl.][]{CIO}.

Sind die Ressourcen vorhanden, ist die Cloud Native Entwicklung von Anwendungen mit Serverless Functions, auch als
Lambda Funtionen bekannt, möglich.