\section{Migrationsansätze}

Im folgenden Kapitel wird auf die unterschiedlichen Migrationsmethoden des Cloud Computing eingegangen.
Durch die unterschiedlichen Servicelevel bedingt gibt es verschiedene Ansätze, wie die Cloud Migration realisierbar ist \cite[Vgl.][S. 226]{Surianarayanan2019},
angefangen mit dem sogenannten \glqq{Lift and Shift}\grqq{}, bis hin zur Entwicklung Cloud nativer Anwendungen \cite[Vgl.][S. 144]{Zhao2014}.

\subsection{Migration zu IaaS (Rehosting)}
Nach Zhao (2014) ist vorallem das Rehosting die vorgeschlagene Strategie für die Migration zu \ac{IaaS} \cite[Vgl.][S. 144]{Zhao2014}.
Umgangssprachlich wird diese Vorgehensweise auch als \glqq{Lift and Shift}\grqq{} bezeichnet \cite[Vgl.][]{NetApp}.
Bei diesen Strategien werden Anwendungen lediglich auf einer anderen Hardwareplattform installiert, die Anwendungsarchitektur bleibt dabei
unverändert. Dieser Ansatz bietet eine schnelle Lösung zur Migration \cite[Vgl.][]{CIO}.

Der wahrscheinlich größte Vorteil der Migration zu \ac{IaaS} ist, dass die Anwendungsarchitektur nicht verändert werden muss und die Migration somit
schnell und ohne großen Aufwand vollzogen werden kann. Ein Nachteil ist dagegen, dass die Migration zu \ac{IaaS} nicht die vollen Möglichkeiten der
Cloud ausnutzt.

\subsection{Migration zu PaaS (Refactoring/Rebuilding)}
Das Refactoring oder Rebuilding von Anwendungen gehört nach Zhao (2014) zu den empfohlenen Strategien für die Migration zu \ac{PaaS} \cite[Vgl.][S. 144]{Zhao2014}.
Rebuilding bedeutet, den bisherigen Code zu re-architekten um die Vorteile der Cloud Plattform nutzen zu können \cite[Vgl.][]{CIO}.

Durch die vom Cloud Provider teils vorgegebene Plattform (z.B. Middleware und Datenbanken) müssen Anwendungen dieser entsprechend angepasst werden \cite[Vgl.][S. 227]{Surianarayanan2019}.
Der Mehrwert einer Migration zu \ac{PaaS} ist, dass der Anwender die IT-Infrastruktur nicht mehr managen muss und somit Aufwände reduziert werden können
\cite[Vgl.][S. 6]{Pahl}.

\subsection{Migration zu SaaS (Cloud Native Entwicklung)}
Der Schritt zu \ac{SaaS} ist nach Zhao (2014) auf verschiedenen Wegen zu erreichen. Demnach sei eine Anwendung durch einen SaaS Service entweder zu ersetzen,
enstprechend der SaaS Prinzipien zu überarbeiten oder zu einem \ac{SaaS} Service unzustrukturieren \cite[Vgl.][S. 144]{Zhao2014}.

Das Augeben einer Anwendung und vollständige Ersetzen dieser durch Nutzung des \ac{SaaS} Modells bedarf eines
größeren Aufwand und ein Entwicklungsteam zur Umsetzung der Anforderungen \cite[Vgl.][]{CIO}.

Sind die Ressourcen vorhanden, ist die Cloud Native Entwicklung von Anwendungen mit Serverless Functions, auch als
Lambda Funtionen bekannt, möglich.