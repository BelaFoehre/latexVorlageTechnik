\section{Serverless Computing} % vs. Server-Aware
\label{sec:serverless-serveraware}

% \subsection{Server-Aware}
% \label{sec:server-aware}

% Der englische Begriff \textit{Server Aware} bedeutet auf deutsch, sich dem Server Bewusst sein. Bei der Nutzung von \ac{IaaS} hat der Entwickler die meiste Kontrolle über Anwendungen und die Infrastruktur in der Cloud und ist verantwortlich für die Bereitstellung von zum Beispiel Hardwareressourcen und \acp{VM} \cite[Vgl.][S. 3]{Baldini2017}.

% \subsection{Serverless}
% \label{sec:serverless}

% Beim Einsatz von \ac{PaaS} und \ac{SaaS} Service Modellen ist der Entwickler ''nichtwissend'' (\textit{unaware}) über die Cloud Infrastruktur \cite[Vgl.][S. 3]{Baldini2017}. \textit{Serverless}, also Serverlos ist eigentlich kein zutreffender Name, da die Infrastruktur und Server nach wie vor existieren, der Entwickler sich lediglich nicht darum kümmern muss, wie diese aussehen \cite[Vgl.][S. 5]{Baldini2017}.

In den vergangenen Jahren hat sich \textit{Serverless Computing} als neues Paradigma des Cloud Computing entwickelt \cite[Vgl.][S. 44]{Castro2019}\cite[Vgl.][S. 64]{Anel2020}. Mit \textit{Serverless Computing} sollen sich die Vorteile der Cloud optimal ausnutzen lassen \cite[Vgl.][S. 6]{Eivy2017}\cite[Vgl.][S. 8]{Jonas2019}. Zu diesen Vorteilen gehört unter anderem die Skalierbarkeit der Infrastruktur \cite[Vgl.][S. 1ff]{Armbrust2009}\cite[Vgl.][S. 234]{Villamizar2017}\cite[Vgl.][S. 884]{Adzic2017}.

\textit{Serverless} Computing bedeutet, dass die Entwickler unabhängig von der Cloud-Infrastruktur entwickeln können. Der Begriff wird noch besser durch den gegenteiligen Begriff \textit{Server Aware} (dt. Server-bewusst) verständlich, da dieser deutlich macht, dass Entwickler sich dabei der Server und Infrastruktur bewusst sein müssen. Bei \textit{Serverless Computing} ist dagegen keine Kenntnis über die zugrundeliegende Infrastruktur notwendig \cite[Vgl.][S. 5]{Jonas2019}\cite[Vgl.][S. 1]{Hellerstein2018}\cite[Vgl.][S. 46]{Castro2019}\cite[Vgl.][S. 64]{Anel2020}.

Darüber hinaus bringt \textit{Serverless} den Vorteil mit sich, dass eine Abrechnung der Kosten nur dann erfolgt, wenn die Funktionen auch ausgeführt werden, da die Ressourcen nur für den Ausführungszeitraum bereitgestellt werden \cite[Vgl.][S. 46]{Castro2019}. Mit \textit{Serverless} kann die Infrastruktur effizienter genutzt werden, da diese nicht dauerhaft für bestimmte Anwendungen reserviert ist und die Produktivität in der Entwicklung steigt \cite[Vgl.][S. 9]{Jonas2019}. Da für diese automatische Skalierung und Bereitstellung der Ressourcen der Cloud Provider verantwortlich ist, kann diese für den Anwender transparent gemacht werden \cite[Vgl.][S. 47]{Castro2019}
\pagebreak