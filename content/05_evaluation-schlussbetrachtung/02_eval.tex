\chapter{Evaluation}
% Ergebinsse der Untersuchung
\section{Vor- und Nachteile der Cloud Migration}
Wie einleitend in dieser Arbeit aufgezeigt, nutzen immer mehr Unternehmen das Cloud Computing in den verschiedensten Unternehmensbereichen. Nachfolgend soll bewertet werden, in welcher Form sich Vor- und Nachteile aus der in dieser Arbeit untersuchten Migration ergeben haben und wie diese sich auf das Gesamtbild der Cloud Migration übertragen lassen.

\subsection{Vorteile}
Aus der beispielhaften Migration des \textit{Collect Service} in die Cloud ergeben sich folgende Vorteile für die Anwendung:

In der Cloud bereitgestellt ist der Service fortan rund um die Uhr verfügbar und unabhängig von der ausführenden Maschine. Es ist nicht mehr notwendig die Anwendung auf einem lokalen Rechner zu installieren (oder einen Rechner zu verwenden, wo diese bereits installiert ist) und jeder Benutzer kann diese nun über eine \ac{API} erreichen.

Updates der Anwendung werden automatisiert über die \ac{CI/CD}-Pipeline deployed und müssen nicht mehr manuell durchgeführt werden. Damit entfällt der Aufwand, vor jeder Nutzung die Aktualität der Anwendung zu überprüfen für alle Anwender.
\pagebreak

\subsection{Nachteile}
Neben den positiven Aspekten die bei der Migration in die Cloud aufgekommen sind, haben sich auch einige Nachteile ergeben:

Das Rebuilding einer Anwendung bringt, bevor die Anwendung fertig entwickelt ist, zuerst einen großen Entwicklungsaufwand mit sich. Ein Entwickler muss die Funktionen der Anwendung verstehen und reproduzieren ohne, dass Funktionen aus der ursprünglichen Anwendung verloren gehen.
\pagebreak