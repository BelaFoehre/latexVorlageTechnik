\chapter{Schlussbetrachtung}
\label{chap:schlussbetrachtung}
In dieser Arbeit wird aufgezeigt, wie sich das Cloud Computing in den vergangenen Jahren entwickelt hat und die Nutzung dessen stetig zunimmt. Anschließend wurden einige Herausforderungen aufgezeigt, die eine Migration einer Anwendung in die Cloud mit sich bringen kann, bevor diese an einem praktischen Beispiel mit der Migration eines Prototypen untersucht wurde. In diesem Kapitel wird noch einmal abschließend auf die untersuchten Aspekte eingegangen und ein Ausblick auf mögliche zukünftige Forschung im Themenumfeld Cloud Computing und Cloud Migration gegeben. \\

% Zusammenfassung der Forschungsergebnisse
\textbf{Zusammenfassung der Forschungsergebnisse}

Cloud Computing hat sich in Europa und auch weltweit in den vergangenen Jahren stark entwickelt und einen immer größere Marktanteil gewonnen. Verbunden mit dieser Entwicklung steht auch die Migration von lokal laufenden Anwendungen in die Cloud, zur Vereinfachung von Prozessen zur Erhöhung der Performance und Zurverfügungstellung von Anwendungen für eine Vielzahl von Anwendern. Eine solche Cloud Migration wurde in der vorliegenden Arbeit anhand einer Anwendung aus dem Projektmanagement untersucht, mit dem Ergebnis, dass sich die Migration für diese Anwendung in der aktuellen Situation, nicht rentiert und ein lokaler Betrieb einiges an Aufwand spart, trotz einiger Vorteile der migrierten Cloud-nativen Anwendung, wie zum Beispiel eine schnellere Bereitstellung der Anwendung oder die Skalierbarkeit der Cloud Infrastruktur. Einige der untersuchten Aspekte werden nachfolgend noch einmal erläutert, um weitere Forschungsmöglichkeiten aus den Erkenntnissen abzuleiten. \\

% Generelle Cloud Adaption
\textbf{Generelle Cloud Adaption}

Die in Kapitel \ref{sec:cloud-computing} beschriebene Entwicklung des Cloud Computing gibt auch einen Ausblick für die Zukunft. So wurde in Kapitel \ref{sec:entwicklung} aufgezeigt, dass auch für 2023 ein Anstieg in der Nutzung der Public Cloud erwartet wird und es zeichnet sich nicht ab, dass der Anstieg über 2023 abflacht. Auch eine \textit{Forrester} Studie zeigt, dass in Deutschland, Frankreich und dem Vereinigtem Königreich bereits ungefähr 92\% der Unternehmen mittlerweile die Cloud nutzen, also Unternehmen, die mindestens ein Organisationsmodell (Private, Public oder Hybrid Cloud) nutzen. Ungefähr 78\% nutzen darüber hinaus mehr als ein Organisationsmodell und ca. 75\% haben mehr als einen Cloud Provider \cite[Vgl.][S. 4]{Rajamani2022}. \pagebreak

% Migration des Collect Service
\textbf{Migration des Collect Service}

Nach erfolgreicher Migration des \textit{Collect Service}, steht einer Migration der übrigen Services aus technischer Sicht nichts im weg. Aus wirtschaftlicher Sicht, sollten jedoch zuvor weitere Untersuchungen angestellt werden, um Finanzielle und Sicherheitsrisiken besser abschätzen zu können. Nachfolgend dazu die in dieser Arbeit gewonnen Erkenntnisse um einen Ausblick geben zu können. \\

% Finanziell
\textbf{Finanziell}

Eine ausführliche finanzielle Betrachtung der Cloud Migration wurde zwar eingangs dieser Arbeit ausgeschlossen, da eine solche Untersuchung nicht Fokus der Forschungsfrage dieser Arbeit war.

Nach oberflächlicher Betrachtung in Kapitel \ref{sec:untersuchung} ist jedoch festzuhalten, dass sich die Cloud Migration, in dem untersuchten Szenario, aus finanzieller Sicht zum aktuellen Zeitpunkt nicht lohnen würde, da der Zeit- und Entwicklungsaufwand zu hoch sind. So lange die Benutzerzahl für die Anwendung nicht über 400 Personen steigt, ist es daher aktuell nicht sinnvoll, diese Anwendung vollständig in die Cloud zu migrieren.

Die in Kapitel \ref{sec:untersuchung} angestellte Kostenkalkulation gibt nur einen groben Überblick und gibt den Anlass, eine detaillierte Kostenanalyse als separate Studie zu dem Thema durchzuführen, bevor über die Durchführung einer vollständigen Migration entschieden wird. \\

% Sicherheit
\textbf{Sicherheit}

Durch eine Betrachtung der Sicherheit der migrierten Anwendung muss festgestellt werden, dass, trotz der Erfüllung der meisten Sicherheitskriterien für die Public Cloud, die realen Anwendungsdaten extrem sensibel und projektkritisch sind. Eine Verarbeitung dieser Daten sollte deshalb besser in einer Private Cloud durchgeführt werden. Sollten Unbefugte Zugang zu den Daten erlangen, wäre die Integrität und Vertraulichkeit nicht mehr gewährleistet.

Bei allgemeiner Betrachtung ist jedoch festzustellen, dass Datenschutz und Datensicherheit auch in einer Public Cloud durchaus erfüllt werden können, sofern die zur Verfügung stehenden Möglichkeiten, wie die Nutzung eines privaten Subnets und eine entsprechende Authentifizierung und Autorisierung innerhalb der Anwendungen, umgesetzt werden. \pagebreak

\textbf{Kritische Reflexion der Forschungsfrage}

Die Forschungsfrage dieser Arbeit war, wie die Migration und Bereitstellung einer ursprünglich lokal laufenden Anwendung in die Cloud umgesetzt werden kann und welche Veränderungen in der Architektur und in Bezug auf die Benutzung vorgenommen werden müssen. Anhand des praktischen Beispiels mit der Migration des \textit{Collect Service} und der Entwicklung eines Prototypen zur Untersuchung der notwendigen Veränderungen an der Anwendungsarchitektur wurde diese Forschungsfrage beantwortet. Die Benutzung der Anwendung verlagerte sich in dem vorliegenden Szenario von einer konsolengesteuerten Anwendung hin zu einer Anwendung mit einer \ac{API}, die zum Beispiel über eine grafische Benutzeroberfläche erreichbar gemacht werden kann. \\

\textbf{Ausblick auf weitere Forschungsmöglichkeiten}

Aus den Forschungsergebnisse haben sich zusammenfassend einige Möglichkeiten ergeben, weitere Forschung zu betreiben. Dazu gehört unter anderem eine ausführliche finanzielle Studie zur Cloud Migration der in dieser Arbeit untersuchten Anwendung, in welcher alle finanziellen Aspekte berücksichtigt werden. Eine solche detaillierte Studie würde weitaus mehr Erkenntnisse zu den finanziellen Vor- und Nachteilen der Cloud Migration gewinnen können.

Darüber hinaus ist ebenfalls bezüglich des Themas Sicherheit personenbezogener oder projektkritischer Daten in der Public Cloud weitere Forschung zu betreiben. \\

\textbf{Ausblick auf das Themenumfeld der Cloud Migration}

Anhand der in dieser Arbeit gewonnenen Erkenntnisse kann abgeleitet werden, dass auch in Zukunft die Migration von lokalen Anwendungen in die Cloud eine größere Rolle für Unternehmen spielen wird. Wie sich auch bei der Untersuchung der finanziellen Aspekte herausstellt, ist es in vielen Fällen nur eine Frage der Zeit, bis sich die Migration im Gegensatz zu einem lokalen Betrieb der Anwendung finanziell rentiert.