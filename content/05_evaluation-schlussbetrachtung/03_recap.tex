\chapter{Schlussbetrachtung}
\label{chap:schlussbetrachtung}
In dieser Arbeit wird aufgezeigt, wie sich das Cloud Computing in den vergangenen Jahren entwickelt hat und die Nutzung dessen stetig zunimmt. Anschließend werden einige Herausforderungen aufgezeigt, die eine Migration von Anwendungen in die Cloud mit sich bringen kann, bevor diese an einem praktischen Beispiel untersucht wurde.

% Generelle Cloud Adaption
\textbf{Generelle Cloud Adaption}
Eine \textit{Forrester} Studie zeigt, dass in Deutschland, Frankreich und dem Vereinigtem Königreich bereits ungefähr 92\% der Unternehmen mittlerweile Cloud-Anwender sind, also Unternehmen, die mindestens ein Organisationsmodell (Private, Public oder Hybrid Cloud) der Cloud nutzen. Ungefähr 78\% nutzen darüber hinaus noch ein weiteres Organisationsmodell und ca. 75\% haben mehr als einen Cloud Provider \cite[Vgl.][S. 4]{Rajamani2022} 

% Finanziell
\textbf{Finanziell}

Eine ausführliche finanzielle Betrachtung der Cloud Migration wurde zwar eingangs dieser Arbeit ausgeschlossen, da diese den Rahmen der Arbeit gesprengt hätte. Oberflächlich ist jedoch festzuhalten, dass sich die Cloud Migration, in dem vorliegenden untersuchten Fall, aus finanzieller Sicht nicht lohnt, da der Zeit- und Entwicklungsaufwand zu hoch ist. So lange die Benutzerzahl für die Anwendung nicht deutlich steigt, ist es daher aktuell nicht sinnvoll, diese Anwendung vollständig in die Cloud zu migrieren.

% Sicherheit
\textbf{Sicherheit}

Mit einer abschließenden Betrachtung der Sicherheit der Daten in der migrierten Anwendung muss festgestellt werden, dass, trotz der Erfüllung der meisten Sicherheitskriterien für die Public Cloud, die realen Anwendungsdaten so sensibel und projektkritisch sind, dass eine Verarbeitung dieser Daten besser in einer Private Cloud durchgeführt werden sollte. Das Projekt in dessen Umfeld die Anwendung eingesetzt wird generiert hohe Umsätze, sollten Unbefugte Kenntnis darüber erlangen wäre Integrität nicht mehr gewährleistet.

Bei allgemeiner Betrachtung ist jedoch festzustellen, dass Datenschutz und Datensicherheit auch in einer Public Cloud durchaus gegeben sind, sofern die zur Verfügung stehenden Möglichkeiten, wie die Nutzung eines privaten Subnetzes und Authentifizierung und Autorisierung innerhalb der Anwendungen, umgesetzt werden.
\pagebreak

% Zusammenfassung der Forschungsergebnisse
\textbf{Zusammenfassung der Forschungsergebnisse}

Abschließend sollen noch einmal alle Forschungsergebnisse dieser Arbeit zusammengefasst werden.

Cloud Computing hat sich in Europa und auch weltweit in den vergangenen Jahren stark entwickelt und einen immer größere Marktanteil gewonnen. Verbunden mit dieser Entwicklung steht auch die Migration von lokal laufenden Anwendungen in die Cloud, zur Vereinfachung von Prozessen zur Erhöhung der Performance und zur Verfügungstellung von Anwendungen für eine Vielzahl von Anwendern. Eine solche Cloud migration wurde in der vorliegenden Arbeit anhand einer Anwendung aus dem Projektmanagement untersucht, mit dem Ergebnis, dass sich die Migration für diese Anwendung in der aktuellen Situation, nicht rentiert und ein lokaler Betrieb einiges an Aufwand spart, trotz einiger Vorteile der migrierten Cloud-nativen Anwendung.